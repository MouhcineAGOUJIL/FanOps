\documentclass[a4paper,12pt]{article}

% Packages pour l'encodage et la langue
\usepackage[utf8]{inputenc}
\usepackage[T1]{fontenc}
\usepackage[french]{babel}
\usepackage{lmodern}
\usepackage{microtype}

% Mise en page
\usepackage[margin=2.5cm]{geometry}
\usepackage{fancyhdr}
\usepackage{lastpage}
\usepackage{titlesec}
\usepackage{float}

% Graphiques et Couleurs
\usepackage{graphicx}
\usepackage{xcolor}
\usepackage{tcolorbox}
\usepackage{tikz}
\usetikzlibrary{shapes,arrows,positioning,shadows}

% Liens et Code
\usepackage{hyperref}
\usepackage{listings}
\usepackage{booktabs}

% Configuration des couleurs
\definecolor{primary}{RGB}{0, 51, 102} % Bleu foncé professionnel
\definecolor{secondary}{RGB}{204, 0, 0} % Rouge (FanOps)
\definecolor{accent}{RGB}{0, 153, 76}   % Vert (Succès/Validé)
\definecolor{codegray}{rgb}{0.95,0.95,0.95}
\definecolor{codeblue}{rgb}{0.0,0.0,0.6}
\definecolor{yamlkey}{rgb}{0.0,0.5,0.0}

% Configuration des listings de code
\lstset{
    backgroundcolor=\color{codegray},
    basicstyle=\ttfamily\footnotesize,
    breaklines=true,
    captionpos=b,
    frame=single,
    keywordstyle=\color{codeblue}\bfseries,
    commentstyle=\color{green!50!black},
    stringstyle=\color{secondary},
    numbers=left,
    numberstyle=\tiny\color{gray},
    stepnumber=1,
    tabsize=2,
    showstringspaces=false
}

\lstdefinelanguage{yaml}{
  keywords={true,false,null,y,n},
  keywordstyle=\color{darkgray}\bfseries,
  basicstyle=\ttfamily\footnotesize,
  sensitive=false,
  comment=[l]{\#},
  morecomment=[s]{/*}{*/},
  commentstyle=\color{purple}\ttfamily,
  stringstyle=\color{blue}\ttfamily,
  moredelim=[l][\color{orange}]{\&},
  moredelim=[l][\color{magenta}]{*},
  moredelim=**[il][\color{secondary}{: }],
  morestring=[b]',
  morestring=[b]"
}

% En-têtes et pieds de page
\pagestyle{fancy}
\fancyhf{}
\lhead{\textbf{Projet FanOps - CAN 2025}}
\rhead{Module M2 : Secure Gates}
\rfoot{Page \thepage \hspace{1pt} sur \pageref{LastPage}}
\renewcommand{\headrulewidth}{1pt}
\renewcommand{\footrulewidth}{1pt}

% Titre du document
\title{
    \vspace{-2cm}
    \begin{tcolorbox}[colback=primary,colframe=primary,arc=0pt,boxrule=0pt, width=\textwidth]
        \centering \textcolor{white}{\Huge \textbf{Rapport Technique Détaillé}} \\
        \vspace{0.5cm}
        \textcolor{white}{\Large \textbf{Service de Sécurité M2 (Secure Gates)}}
    \end{tcolorbox}
    \vspace{2cm}
    \begin{center}
        \Huge \textbf{Architecture Cloud Hybride \& Mise en Œuvre DevSecOps}
    \end{center}
    \vspace{2cm}
    \large
    \textbf{Contexte :} Coupe d'Afrique des Nations 2025 \\
    \textbf{Domaine :} Cloud Computing, Serverless, SIEM
}
\author{\textbf{Équipe Technique FanOps}}
\date{\today}

\begin{document}

\maketitle
\thispagestyle{empty}
\newpage

\tableofcontents
\newpage

% ---------------------------------------------------------------------------
\section{1. Contexte et Enjeux}

L'organisation de la \textbf{Coupe d'Afrique des Nations (CAN) 2025} au Maroc représente un défi logistique et sécuritaire majeur. La gestion des accès aux stades pour des millions de supporters nécessite une infrastructure infaillible.

Le module \textbf{M2 - Secure Gates} a pour mission de garantir :
\begin{itemize}
    \item \textbf{L'Authenticité} : Chaque billet doit être cryptographiquement vérifiable.
    \item \textbf{L'Unicité} : Un billet ne peut être utilisé qu'une seule fois (Protection Anti-Replay).
    \item \textbf{La Disponibilité} : Le système doit supporter des pics de charge massifs (100k requêtes/minute avant un match).
    \item \textbf{L'Observabilité} : Toute tentative de fraude doit être détectée en temps réel.
\end{itemize}

% ---------------------------------------------------------------------------
\section{2. Architecture Cloud Hybride}

Nous avons opté pour une stratégie \textbf{"Best of Breed"}, combinant la puissance de calcul d'AWS avec l'intelligence de sécurité de Microsoft Azure.

\subsection{2.1 Diagramme d'Architecture}

\begin{center}
\begin{tikzpicture}[node distance=1.5cm, auto, >=stealth]
    % Styles
    \tikzstyle{service} = [rectangle, draw=primary, fill=blue!5, text width=2.5cm, text centered, rounded corners, minimum height=1.2cm]
    \tikzstyle{db} = [cylinder, draw=primary, fill=orange!10, shape border rotate=90, aspect=0.25, text width=2cm, text centered, minimum height=1.5cm]
    \tikzstyle{cloud} = [draw, ellipse, fill=gray!5, node distance=3cm, minimum height=2em]

    % Nodes
    \node[service, fill=green!10] (client) {Frontend React (Vite)};
    \node[service, below=of client] (api) {AWS API Gateway};
    \node[service, below=of api] (lambda) {AWS Lambda (Node.js)};
    
    \node[db, below left=of lambda] (dynamo) {DynamoDB (Tickets/Users)};
    \node[service, below right=of lambda] (kms) {AWS KMS};
    
    \node[service, right=of api, xshift=3cm, fill=blue!10] (sentinel) {Azure Sentinel};
    \node[service, below=of sentinel] (ec2) {EC2 (ZAP Scanner)};

    % Edges
    \draw[->, thick] (client) -- node[right] {HTTPS/REST} (api);
    \draw[->, thick] (api) -- (lambda);
    \draw[->, thick] (lambda) -- (dynamo);
    \draw[->, thick] (lambda) -- (kms);
    \draw[->, thick, dashed, color=red] (ec2) -- node[above] {Attaques} (api);
    \draw[->, thick, dashed, color=blue] (lambda) -- node[above] {Logs} (sentinel);
    \draw[->, thick, dashed, color=blue] (ec2) -- node[right] {Rapports S3} (sentinel);

\end{tikzpicture}
\end{center}

% ---------------------------------------------------------------------------
\section{3. Cycle de Mise en Œuvre (DevSecOps)}

Cette section détaille le processus complet, de l'écriture du code à son exécution en production.

\subsection{3.1 Développement (Code)}
Le backend est développé en \textbf{Node.js 20.x}. Nous avons adopté une architecture modulaire pour garantir la maintenabilité.

\begin{itemize}
    \item \textbf{Structure du Projet} :
    \begin{itemize}
        \item \texttt{src/handlers/} : Contient les points d'entrée Lambda (ex: \texttt{verifyTicket.js}). Chaque fonction est isolée et stateless.
        \item \texttt{src/services/} : Logique métier (ex: \texttt{ticketService.js}).
        \item \texttt{src/utils/} : Utilitaires transverses (ex: \texttt{kmsHelper.js} pour le déchiffrement).
    \end{itemize}
    \item \textbf{Sécurité dans le Code} :
    \begin{itemize}
        \item Utilisation de \texttt{bcrypt} pour le hachage des mots de passe.
        \item Validation stricte des entrées (Input Validation) pour prévenir les injections NoSQL.
        \item Gestion des erreurs centralisée pour ne jamais exposer de stack traces à l'utilisateur.
    \end{itemize}
\end{itemize}

\subsection{3.2 Configuration (Infrastructure as Code)}
Toute l'infrastructure est définie dans le fichier \texttt{serverless.yml}. Cela permet des déploiements reproductibles et auditable.

\begin{lstlisting}[language=yaml, caption=Extrait de la configuration Serverless]
service: can2025-secure-gates
provider:
  name: aws
  runtime: nodejs20.x
  region: eu-west-1
  iamRoleStatements:
    - Effect: Allow
      Action:
        - dynamodb:GetItem
        - dynamodb:PutItem
      Resource: !GetAtt UsedJtiTable.Arn

functions:
  verifyTicket:
    handler: src/handlers/verifyTicket.handler
    events:
      - http:
          path: /security/verifyTicket
          method: post
          cors: true

resources:
  Resources:
    UsedJtiTable:
      Type: AWS::DynamoDB::Table
      Properties:
        BillingMode: PAY_PER_REQUEST
        TimeToLiveSpecification:
          AttributeName: ttl
          Enabled: true
\end{lstlisting}

\textbf{Points Clés de la Configuration :}
\begin{itemize}
    \item \textbf{IAM Roles (Moindre Privilège)} : Chaque fonction ne reçoit que les permissions strictement nécessaires (ex: \texttt{verifyTicket} peut lire \texttt{SoldTickets} mais pas \texttt{Users}).
    \item \textbf{Variables d'Environnement} : Les noms de tables et les configurations sont injectés dynamiquement selon le stage (\texttt{dev}, \texttt{prod}).
\end{itemize}

\subsection{3.3 Déploiement (CI/CD)}
Le déploiement est automatisé via le Framework Serverless, qui orchestre CloudFormation sous le capot.

\textbf{Étapes du Déploiement :}
\begin{enumerate}
    \item \textbf{Packaging} : Le code est zippé, excluant les fichiers de développement (\texttt{devDependencies}).
    \item \textbf{Upload} : Le zip est uploadé sur un bucket S3 de déploiement.
    \item \textbf{CloudFormation} :
    \begin{itemize}
        \item Création/Mise à jour de la Stack.
        \item Provisionnement des tables DynamoDB.
        \item Création des Rôles IAM.
        \item Configuration de l'API Gateway.
    \end{itemize}
    \item \textbf{Mise à jour Lambda} : Le code des fonctions est mis à jour.
\end{enumerate}

\begin{lstlisting}[language=bash, caption=Commande de deploiement]
# Deploiement sur l'environnement de developpement
serverless deploy --stage dev --region eu-west-1
\end{lstlisting}

\subsection{3.4 Fonctionnement Opérationnel}
Une fois en production, voici le cycle de vie d'une requête :

\begin{enumerate}
    \item \textbf{Entrée} : L'API Gateway reçoit une requête HTTPS. Elle vérifie le quota (Throttling) et l'authentification.
    \item \textbf{Exécution (Cold/Warm Start)} :
    \begin{itemize}
        \item Si un conteneur Lambda est chaud, il est réutilisé (latence < 10ms).
        \item Sinon, AWS provisionne un nouveau conteneur (Cold Start ~200ms).
    \end{itemize}
    \item \textbf{Logique} : La fonction exécute la vérification (KMS, DynamoDB).
    \item \textbf{Sortie} : La réponse JSON est renvoyée.
    \item \textbf{Observabilité} :
    \begin{itemize}
        \item Les logs sont poussés asynchronement vers CloudWatch.
        \item La Lambda \texttt{sentinelShipper} se réveille pour transférer ces logs vers Azure Sentinel.
    \end{itemize}
\end{enumerate}

% ---------------------------------------------------------------------------
\section{4. Sécurité Offensive Automatisée (IaaS)}

Pour garantir une posture de sécurité proactive, nous avons implémenté une chaîne d'audit continu.

\subsection{4.1 Infrastructure de Test}
Une instance \textbf{AWS EC2 (Ubuntu 22.04)} agit comme un "attaquant interne".
\begin{itemize}
    \item \textbf{Rôle IAM} : \texttt{SecurityInstanceRole}. Principe du moindre privilège : cette instance ne peut qu'écrire dans le bucket S3 des rapports, elle n'a aucun accès à DynamoDB ou Lambda.
\end{itemize}

\subsection{4.2 Automatisation avec OWASP ZAP}
Un script Bash, exécuté quotidiennement par \textbf{Cron} (02h00 AM), orchestre les tests.

\begin{lstlisting}[language=bash, caption=Extrait du script d'automatisation ZAP]
#!/bin/bash
TARGET="https://api.can2025.com/dev"
REPORT="/tmp/zap-report-$(date +%F).html"

# 1. Lancement du scan de vulnerabilites (SQLi, XSS, etc.)
/opt/zap/zap.sh -cmd -quickurl $TARGET -quickout $REPORT

# 2. Exfiltration du rapport vers S3 (Securise)
aws s3 cp $REPORT s3://can2025-security-reports/
\end{lstlisting}

Ce mécanisme nous permet de recevoir chaque matin un rapport de sécurité frais, identifiant toute nouvelle vulnérabilité introduite par un déploiement récent.

% ---------------------------------------------------------------------------
\section{5. Intégration SIEM (Azure Sentinel)}

La surveillance est centralisée dans Azure pour bénéficier des capacités d'IA de Microsoft.

\subsection{5.1 Flux de Données}
\begin{enumerate}
    \item \textbf{Logs Applicatifs} : Lambda envoie ses logs (erreurs, latence) à CloudWatch.
    \item \textbf{Forwarder} : La Lambda \texttt{sentinelShipper} s'abonne aux Log Groups CloudWatch.
    \item \textbf{Ingestion} : Elle formate les logs en JSON et les pousse vers l'API \textbf{Azure Log Analytics Data Collector}.
\end{enumerate}

\subsection{5.2 Détection des Menaces (KQL)}
Nous utilisons le langage Kusto (KQL) pour créer des règles d'alerte.

\textbf{Exemple : Détection d'attaque Brute Force}
\begin{lstlisting}[language=SQL, caption=Requete KQL Sentinel]
AppEvents
| where Name == "Login_Failure"
| summarize FailureCount = count() by IPAddress, bin(TimeGenerated, 5m)
| where FailureCount > 10
| project TimeGenerated, IPAddress, FailureCount, "Brute Force Detected"
\end{lstlisting}

% ---------------------------------------------------------------------------
\section{6. Analyse FinOps (Coûts)}

L'architecture Serverless permet une optimisation drastique des coûts.

\begin{table}[H]
\centering
\begin{tabular}{|l|l|r|}
\hline
\textbf{Service} & \textbf{Métrique} & \textbf{Coût Est. / Mois} \\ \hline
AWS Lambda & 1M requêtes (128MB) & \$0.20 \\ \hline
API Gateway & 1M requêtes & \$3.50 \\ \hline
DynamoDB & 1M écritures & \$1.25 \\ \hline
EC2 (t3.micro) & Instance Réservée & \$4.00 \\ \hline
Azure Sentinel & 5 GB logs/mois & \$12.00 \\ \hline
\textbf{Total} & & \textbf{\textasciitilde \$21.00} \\ \hline
\end{tabular}
\caption{Estimation des coûts pour une charge moyenne}
\end{table}

% ---------------------------------------------------------------------------
\section{7. Conclusion}

Le module \textbf{M2 - Secure Gates} est une démonstration concrète d'une architecture Cloud moderne, sécurisée et résiliente. 

En intégrant la sécurité à tous les niveaux (Chiffrement KMS, Anti-Replay DynamoDB, Audit ZAP, Surveillance Sentinel), nous avons construit une forteresse numérique capable de protéger l'intégrité de la CAN 2025 tout en restant flexible et économique.

\end{document}
