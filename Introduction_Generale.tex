\chapter*{Introduction Générale}
\addcontentsline{toc}{chapter}{Introduction Générale}

\section*{Contexte Stratégique : CAN 2025}
L'organisation de la Coupe d'Afrique des Nations (CAN) 2025 au Maroc représente un défi logistique et sécuritaire majeur. Avec des millions de spectateurs attendus et une exposition médiatique mondiale, l'infrastructure technologique des stades doit garantir une fluidité parfaite et une sécurité sans faille. C'est dans ce contexte que naît \textbf{FanOps}.

\section*{Vision du Projet}
\textbf{FanOps} est une plateforme Cloud native et distribuée, conçue pour orchestrer les opérations critiques des stades en temps réel. Elle ne se contente pas de gérer des billets ; elle fusionne la sécurité, l'intelligence artificielle et la supervision pour offrir une expérience "Smart Stadium".

Notre approche repose sur une architecture \textbf{Multi-Cloud Hybride} ("Best of Breed"), tirant parti des forces spécifiques de chaque fournisseur :
\begin{itemize}
    \item \textbf{Azure} pour l'orchestration des flux et le SIEM.
    \item \textbf{AWS} pour la sécurité serverless et le compute critique.
    \item \textbf{GCP} pour l'intelligence artificielle et le marketing prédictif.
\end{itemize}

\section*{Les Piliers de FanOps}
Le projet s'articule autour de quatre modules interconnectés :

\subsection*{1. M1 - Flow Controller (Gestion des Flux)}
Hébergé sur \textbf{Azure}, c'est le système nerveux central qui orchestre l'expérience fan. Il gère l'information en temps réel pour fluidifier les accès et éviter les goulots d'étranglement aux abords des stades.

\subsection*{2. M2 - Secure Gates (Sécurité \& Contrôle d'Accès)}
Hébergé sur \textbf{AWS}, ce module est la forteresse numérique du projet.
\begin{itemize}
    \item \textbf{Architecture Serverless} : Utilisation de Lambda et DynamoDB pour une mise à l'échelle infinie.
    \item \textbf{Sécurité Offensive} : Intégration de scans automatisés (OWASP ZAP) et architecture Zero Trust.
    \item \textbf{Anti-Fraude} : Détection instantanée des "Replay Attacks" et validation cryptographique (JWT).
\end{itemize}

\subsection*{3. M3 - Forecast AI (Intelligence Prédictive)}
Un module d'Intelligence Artificielle (hébergé sur \textbf{AWS}) qui anticipe l'avenir. En analysant les données historiques et contextuelles (équipes, horaire, enjeu), ce service prédit l'affluence exacte des matchs, permettant d'ajuster les ressources de sécurité de manière proactive.

\subsection*{4. M4 - Sponsor AI (Intelligence Marketing)}
Hébergé sur \textbf{Google Cloud Platform (GCP)}, ce module révolutionne la monétisation.
\begin{itemize}
    \item \textbf{IA Contextuelle} : Analyse en temps réel le contexte du match (score, météo, minute).
    \item \textbf{Publicité Dynamique} : Recommande le sponsor le plus pertinent (ex: boisson fraîche si canicule).
    \item \textbf{Hyper-Personnalisation} : Maximise le ROI des sponsors.
\end{itemize}

\section*{Une Approche DevSecOps}
FanOps incarne une méthodologie moderne : sécurité dès la conception ("Security by Design"), déploiement continu (CI/CD) et surveillance centralisée via \textbf{Azure Sentinel}. C'est la convergence du Cloud Computing et de l'événementiel sportif.
